\documentclass[times, 12pt, utf8]{book}
\RequirePackage[utf8]{inputenc}
\RequirePackage[croatian]{babel}
\usepackage[utf8]{inputenc}
\usepackage{amsmath}
\usepackage{multirow}
\usepackage{alltt}
\usepackage{listings}
\usepackage{color}
\usepackage{array}
\usepackage{hyperref}
\usepackage{verbatim}
\usepackage[a4paper, top=1.5cm, bottom=1.5cm, left=1.5cm, right=1.5cm]{geometry}
\pagestyle{plain}

\usepackage{titlesec}
\titleformat{\chapter}{}{\thechapter}{.01em}{}

\setlength{\parindent}{0pt}
\setlength{\parskip}{1ex plus 0.5ex minus 0.2ex}

\newenvironment{myindentpar}[1]%
{\begin{list}{}%
         {\setlength{\leftmargin}{#1}}%
         \item[]%
}
{\end{list}}

\addto\captionscroatian{%
  \renewcommand{\contentsname}%
    {}%
}

\titlespacing*{\chapter}{0pt}{50pt}{-60pt}

\title{
\vfill
\textbf{Prevođenje programskih jezika}\\
\vspace{30pt}
Informacije o predmetu za akademsku godinu 2015./2016. \\
\vspace{40pt}
}
\author{
\emph{Predavanja:} \\
Prof.~dr.~sc.~Siniša Srbljić (nositelj) \\
Doc. dr.~sc.~Dejan Škvorc \\
Doc. dr.~sc.~Ante Đerek \\
Dr.~sc.~Goran Delač \\
Dr.~sc.~Marin Šilić \\
\vspace{30pt}
Dr.~sc.~Klemo Vladimir \\
\emph{Auditorne i laboratorijske vježbe:} \\
Dr.~sc.~Goran Delač \\
Dr.~sc.~Marin Šilić \\
Dr.~sc.~Klemo Vladimir \\
\vspace{30pt}
Zvonimir Pavlić, mag.~ing. \\
\emph{Službena web-stranica predmeta:} \\
\vspace{30pt}
\url{http://www.fer.hr/predmet/ppj_a} \\
\emph{Službena e-mail adresa:} \\
\tt{\href{mailto:ppj@zemris.fer.hr}{ppj@zemris.fer.hr}}
}
\date{
\vspace{30pt} 
\emph{Datum posljednje izmjene:} \\
\today \\
\vfill
Fakultet elektrotehnike i računarstva \\
Zavod za elektroniku, mikroelektroniku, inteligentne i računalne sustave
}
\begin{document}
\maketitle

\section*{Informacije o predmetu}

Ovaj dokument sadrži osnovne informacije o organizaciji i provođenju predmeta Prevođenje programskih jezika tijekom akademske godine 2015./2016. 
Molimo studente da se upoznaju s informacijama u ovom dokumentu.

\let\cleardoublepage\relax
\phantomsection

\tableofcontents

\cleardoublepage  
\phantomsection  
\addcontentsline{toc}{section}{Literatura}  
\subsection*{Literatura}

\textbf{Osnovna literatura}

Srbljić: \textit{Prevođenje programskih jezika}, Udžbenik Sveučilišta u Zagrebu, Element, Zagreb, prvo izdanje 2007.

Srbljić: \textit{Jezični procesori 2: Analiza izvornog i sinteza ciljnog programa}, Udžbenik Sveučilišta u Zagrebu, Element, Zagreb, drugo izdanje 2003.

\textbf{Dodatna literatura}

Aho, Lam, Sethi, Ullman: \textit{Compilers: Principles, Techniques and Tools (Dragon book)}, Addison-Wesley, 2006. 

Appel, Palsberg: \textit{Modern Compiler Implementation in Java (Tiger book)}, Cambridge University Press, 2002.

Muchnick: \textit{Advanced Compiler Design and Implementation (Whale book)}, Morgan Kaufman Publishers, 1997.

Cooper, Torczon: \textit{Engineering a Compiler (Ark book)}, Morgan Kaufman Publishers, 2011.

Allen, Kennedy: \textit{Optimizing Compilers for Modern Architectures}, Morgan Kaufman Publishers, 2001.

Scott: \textit{Programming Languages Pragmatics}, Morgan Kaufmann Publishers, 2009.

\cleardoublepage  
\phantomsection
\addcontentsline{toc}{section}{Konzultacije i komunikacija s nastavnicima}
\subsection*{Konzultacije i komunikacija s nastavnicima}

Studente potičemo da komuniciraju s nastavnicima i rješavaju moguće probleme ili putem konzultacija ili slanjem poruke na službenu e-mail adresu predmeta.

Za konzultacije ne postoji zaseban termin, nego se održavaju po potrebi.
Konzultacije su moguće isključivo uz najavu barem jedan dan unaprijed s kratkim opisom problema na službenu e-mail adresu predmeta.

Nadalje, molimo studente da pitanja u vezi predmeta šalju isključivo na službenu e-mail adresu predmeta, a ne pojedinačno nastavnicima. 
Na poruke poslane izravno na e-mail adresu nekog od nastavnika u načelu se neće odgovarati.
Osim toga, na poruke s pitanjima na koje je odgovoreno u ovim uputama ili u obavijestima na službenoj web-stranici predmeta također neće biti poslani odgovori.

\begin{comment}
\cleardoublepage  
\phantomsection  
\addcontentsline{toc}{section}{Administracija bodova}
\subsection*{Administracija}

Za unos i pregled obaveza na predmetu i svih bodova koristi se informacijski sustav \href{http://ferko.fer.hr}{FERKO}.
Molimo studente da porukom na službenu e-mail adresu predmeta na vrijeme dojave moguće pogreške u unosu bodova.

\end{comment}

\cleardoublepage  
\phantomsection  
\addcontentsline{toc}{section}{Red predavanja}
\subsection*{Red predavanja}

Predavanja se u pravilu održavaju prema \href{http://web.zpr.fer.hr/satnica1/predmeti.htm}{službenoj satnici Fakulteta} i kalendaru nastave, sa po dva dvosatna predavanja tjedno.
Iznimno, između nekih predavanja će doći do pauze zbog održavanja auditornih vježbi u terminima predavanja ili zbog neradnih dana.

\textbf{01.~Predavanje} 
\begin{myindentpar}{30pt}
Osnovne faze procesa prevođenja: analiza i sinteza; Analiza izvornog programa: leksička, sintaksna i semantička analiza; Sinteza ciljnog programa: generiranje međukôda. (udžbenik, str.~1-15)
\end{myindentpar}

\textbf{02.~Predavanje} 
\begin{myindentpar}{30pt}
Sinteza ciljnog programa: optimiranje i generiranje ciljnog programa; Nadziranje pogrešaka tijekom prevođenja; Ocjena uspješnosti procesa prevođenja; Razredba jezičnih prevoditelja; (udžbenik, str.~15-31) 
\end{myindentpar}

\textbf{03.~Predavanje}
\begin{myindentpar}{30pt}
Leksička analiza; Uloga i podatkovna struktura leksičkog analizatora; Suradnja sa sintaksnim analizatorom; Nejednoznačnost u leksičkoj analizi; Prilagodba zapisa znakova, leksičke jedinke, leksičke pogreške i postupci oporavka od pogreške; (udžbenik, str.~44-55)
\end{myindentpar}

\textbf{04.~Predavanje}
\begin{myindentpar}{30pt}
Generatori leksičkog analizatora; Program Lex; (udžbenik, str.~56-70)
\end{myindentpar}

\textbf{05.~Predavanje}
\begin{myindentpar}{30pt}
Sintaksna analiza; Uloga i podatkovna struktura sintaksnog analizatora; Jezici za definiranje sintaksnih pravila; Jednostavni postupci parsiranja: parsiranje Co-No tablicom; (udžbenik, str.~71-84)
\end{myindentpar}

\textbf{06.~Predavanje}
\begin{myindentpar}{30pt}
Parsiranje od vrha prema dnu; S, Q i LL(1) gramatika; Određivanje PRIMIJENI skupova; (udžbenik, str.~84-103) 
\end{myindentpar}

\textbf{07.~Predavanje}
\begin{myindentpar}{30pt}
Određivanje PRIMIJENI skupova (nastavak); Prilagodba produkcija LL(1)-gramatici; Nadziranje pogrešaka; Parsiranje od dna prema vrhu; (udžbenik, str.~103-121)
\end{myindentpar}

\textbf{08.~Predavanje}
\begin{myindentpar}{30pt}
Tehnike parsiranja od dna prema vrhu: Pomakni-Pronađi, Pomakni-Reduciraj i prednost operatora; (udžbenik, str.~121-137) 
\end{myindentpar}

\textbf{09.~Predavanje}
\begin{myindentpar}{30pt}
LR parsiranje; (udžbenik, str.~138-147)
\end{myindentpar}

\textbf{10.~Predavanje}
\begin{myindentpar}{30pt}
LR parsiranje (nastavak); Generatori parsera; Program Yacc; Semantička analiza; Uloga i formalni modeli semantičkog analizatora; Zadaci semantičkog analizatora; (udžbenik, str.~147-166) 
\end{myindentpar}

\textbf{11.~Predavanje}
\begin{myindentpar}{30pt}
Zadaci semantičkog analizatora (nastavak); Sintaksom upravljana semantička analiza: atributna prijevodna gramatika; (udžbenik, str.~167-180)
\end{myindentpar}

\textbf{12.~Predavanje}
\begin{myindentpar}{30pt}
L-atributna prijevodna gramatika; Potisni automat za L-atributnu prijevodnu gramatiku; (udžbenik, str.~180-190)
\end{myindentpar}

\textbf{13.~Predavanje}
\begin{myindentpar}{30pt}
Potisni automat za L-atributnu prijevodnu gramatiku (nastavak); Metoda rekurzivnog spusta za L-atributnu prijevodnu gramatiku; Sustav obilježja; (udžbenik, str.~190-200) 
\end{myindentpar}

\textbf{14.~Predavanje}
\begin{myindentpar}{30pt}
Provjera vrijednosti obilježja; Jednakost vrijednosti obilježja; Potpora izvođenju ciljnog programa; (udžbenik, str.~200-223)
\end{myindentpar}

\textbf{15.~Predavanje}
\begin{myindentpar}{30pt}
Apstraktni tipovi podataka izvornog jezika i podatkovni objekti ciljnog jezika; Tijek izvođenja programa zasnovanog na procedurama; Organizacija i postupci dodjele memorije; (udžbenik, str.~223-233) 
\end{myindentpar}

\textbf{16.~Predavanje}
\begin{myindentpar}{30pt}
Pristup nelokalnim imenima; (udžbenik, str.~233-242)
\end{myindentpar}

\textbf{17.~Predavanje}
\begin{myindentpar}{30pt}
Razmjena ulazno/izlaznih parametara procedura; Generiranje međukoda; Razine i oblici međukôda; (udžbenik, str.~243-256) 
\end{myindentpar}

\textbf{18.~Predavanje}
\begin{myindentpar}{30pt}
Razine i oblici međukôda (nastavak); Sintaksom vođeno generiranje međukôda; Generiranje ciljnog programa; Struktura generatora ciljnog programa; (udžbenik, str.~256-265)
\end{myindentpar}

\textbf{19.~Predavanje}
\begin{myindentpar}{30pt}
Struktura generatora ciljnog programa (nastavak); (udžbenik, str.~265-276) 
\end{myindentpar}

\textbf{20.~Predavanje}
\begin{myindentpar}{30pt}
Algoritmi generatora ciljnog programa; Generatori generatora ciljnog programa; (udžbenik, str.~276-285) 
\end{myindentpar}

\textbf{21.~Predavanje}
\begin{myindentpar}{30pt}
Priprema ciljnog programa za izvođenje; Spremi-i-pokreni jezični procesori; Generatori izvodivog i premjestivog ciljnog programa; Generatori zasebnih dijelova programa; Program punitelj i program povezivač; Optimiranje; Analiza izvođenja programa; (udžbenik, str.~286-297)
\end{myindentpar}

\textbf{22.~Predavanje}
\begin{myindentpar}{30pt}
Analiza izvođenja programa (nastavak); (udžbenik, str. 297-305)
\end{myindentpar}

\textbf{23.~Predavanje}
\begin{myindentpar}{30pt}
Strojno zavisno i strojno nezavisno optimiranje; Jednostavni postupak optimiranja zasnovan na prozorčiću; Pregled ostalih postupaka optimiranja; (udžbenik, str.~305-317)
\end{myindentpar}

\cleardoublepage  
\phantomsection  
\addcontentsline{toc}{section}{Auditorne vježbe}
\subsection*{Auditorne vježbe}

Tijekom semestra, u terminima predavanja održavaju se i auditorne vježbe.
Auditorne vježbe namijenjene su za pripremu za međuispite i završni ispit, a u terminima auditornih vježbi rješavaju se primjeri tipičnih problemskih zadataka iz međuispita.

\textbf{I.~auditorne vježbe} \\
Tjedan održavanja: 16.~11.~2015.~–- 20.~11.~2015. \\
Gradivo: Leksička analiza, sintaksna analiza i dio semantičke analize

\textbf{II.~auditorne vježbe} \\
Tjedan održavanja: -  25.~01.~2016.~–- 29.~01.~2016. \\
Gradivo: dio semantičke analize, potpora izvođenju, generiranje i optimiranje ciljnog programa


\cleardoublepage  
\phantomsection  
\addcontentsline{toc}{section}{Međuispiti}
\subsection*{Međuispiti}

Tijekom semestra provode se pismene provjere znanja: međuispit i završni ispit.
Trajanje međuispita i završnog ispita je 120 minuta.
Međuispit i završni ispit sastoje se od 10 zadataka (5 teoretskih pitanja i 5 problemskih zadataka).

\textbf{Međuispit} \\
Tjedan održavanja:  23.~11.~2015.~–-~4.~12.~2015. \\
Udio u ukupnoj ocjeni:  KM = 30 bodova \\
Gradivo: Leksička, sintaksna i semantička analiza

\textbf{Završni ispit} \\
Tjedan održavanja:  1.~2.~2016.~–-~12.~2.~2016. \\
Udio u ukupnoj ocjeni: KZ = 30 bodova \\
Gradivo: Ukupno gradivo ispredavano na predavanjima predmeta, odnosno ukupno gradivo udžbenika.

\cleardoublepage  
\phantomsection  
\addcontentsline{toc}{section}{Ispiti}
\subsection*{Ispiti}

Studenti koji ne polože predmet tijekom semestra putem međuispita i završnih ispita imaju pravo pristupa ispitnim rokovima ako su tijekom semestra ostvarili barem 50\% bodova iz laboratorijskih vježbi (10 bodova za lakšu, a 20 za težu inačicu).

Ispitni rokovi sastoje se od 10 zadataka (5 teoretskih pitanja, 5 problemskih zadataka) i njihov udio u ukupnoj ocjeni je I = 100 bodova.
Drugim riječima, uspjeh na predmetu određen je uspjehom na ispitu.



\cleardoublepage  
\phantomsection  
\addcontentsline{toc}{section}{Uvidi u rezultate pismenih provjera znanja}
\subsection*{Uvidi u rezultate pismenih provjera znanja}

Studenti imaju pravo uvida u rezultate pismenih provjera znanja, odnosno međuispita, završnog ispita, kratkih provjera znanja i ispita na ispitnim rokovima.
Nakon ispravljanja međuispita, završnih ispita i ispita na ispitnim rokovima na službenu web-stranicu predmeta bit će postavljena obavijest o službenom terminu uvida u rezultate.
Uvidi u rezultate bit će postavljeni u termin koji većini studenata stvara najmanje kolizija s ostalim fakultetskim obavezama, a uvidi će uvijek biti na ZEMRIS-u (3.~kat D zgrade).
Mole se studenti da ne ulaze u Zavod, nego da pričekaju ispred južnog ulaza Zavoda.
Studenti koji nisu u mogućnosti doći na uvid, ali isključivo iz opravdanog razloga, trebaju se porukom na službenu e-mail adresu predmeta najaviti s terminom u kojemu mogu doći na uvid, ali barem jedan dan prije odabranog termina za uvid.
Uvidi u rezultate kratkih provjera znanja mogući su također isključivo uz prethodnu najavu na službenu e-mail adresu predmeta, barem jedan dan prije predloženog termina.
Uvidi u rezultate izvan službenog termina za uvide ili bez prethodne najave na službenu e-mail adresu nisu mogući.

\cleardoublepage  
\phantomsection  
\addcontentsline{toc}{section}{Laboratorijske vježbe}
\subsection*{Laboratorijske vježbe}

Studenti na početku semestra trebaju izabrati između jednostavnije i složenije inačice laboratorijskih vježbi.

\subsubsection{Jednostavnija inačica laboratorijskih vježbi}
Udio u ukupnoj ocjeni: L1 = 20 bodova \\
Prag za prolaz = 10 bodova

Jednostavnija inačica laboratorijskih vježbi radi se individualno.
Studenti rješavaju 4 međusobno nezavisna programska zadatka.
Zadaci se predaju i evaluiraju na sustavu SPRUT.

\textbf{I.~laboratorijska cjelina} \\
Sadržaj: Leksički analizator \\
Rok za predaju rješenja: 1.~11.~2015.

\textbf{II.~laboratorijska cjelina} \\
Sadržaj: Sintaksni analizator \\
Rok za predaju rješenja: 22.~11.~2015.

\textbf{III.~laboratorijska cjelina } \\
Sadržaj: Semantički analizator \\
Rok za predaju rješenja: 20.~12.~2015.

\textbf{IV.~laboratorijska cjelina} \\
Sadržaj: Generator ciljnog programa. Optimiranje \\
Rok za predaju rješenja: 24.~1.~2016.


\subsubsection{Složenija inačica laboratorijskih vježbi}
Udio u ukupnoj ocjeni: L2 = 40 bodova \\
Prag za prolaz = 20 bodova

Složenija inačica laboratorijskih vježbi radi se u grupama od 1-4 studenata.
Studenti tijekom vježbi izrađuju vlastiti jezični procesor prema posebnim uputama za laboratorijske vježbe koje će biti objavljene na službenoj web-stranici predmeta.
Studenti samostalno dogovaraju podjelu posla unutar grupe, ali svaki član grupe mora biti upoznat sa svim dijelovima jezičnog procesora.
Kao rezultat laboratorijskih vježbi studenti pokazuju rad svog jezičnog procesora.

Na početku semestra studenti prema vlastitom dogovoru formiraju grupe.
Nakon formiranja grupe, svi članovi iste dojavljuju na e-mail predmeta JMBAG, ime i prezime svakog člana čime potvrđuju da žele raditi zahtjevniju inačicu laboratorijskih vježbi i da žele sudjelovati u navedenoj grupi.
U grupi treba biti točno jedan član koji će biti zadužen za predaju rješenja pojedinih laboratorijskih vježbi na sustav SPRUT i taj član treba biti istaknut u e-mailu.
Rok za formiranje grupe je \textbf{ponedjeljak, 6.~listopada}.
Za studente koji do navedenog roka ne dojave popis članova grupe pretpostavit će se da žele raditi jednostavniju inačicu laboratijskih vježbi.

Laboratorijske vježbe provode se kroz četiri ciklusa, isto kao u jednostavnijoj inačici, a ocjenjivanje se provodi kombinacijom usmenog ispitivanja i računalne evaluacije studentskih rješenja.
Postoje četiri predaje laboratorijskih vježbi putem računala te dva usmena ispitivanja.
Na prvom usmenom ispitivanju ispituju se prva dva ciklusa vježbi, a na drugom ispitivanju preostala dva ciklusa.

Za predaje putem računala koristi se sustav SPRUT.
Predano rješenje evaluira se automatizirano primjenom skupa ispitnih primjera.
Na osnovi evaluacije, svaka predaja boduje se s najviše 10 bodova.
Grupama koje ne predaju programsko ostvarenje rješenja na vrijeme ili nisu zadovoljne ostvarenim rezultatom, omogućit će se \textbf{dodatna predaja} laboratorijske vježbe tjedan dana nakon roka redovne predaje koja će se bodovati s \textbf{najviše 5 bodova, tj.~50\%} bodova koje je moguće ostvariti prvom predajom.
Svi članovi grupe dobivaju ostvareni broj bodova iz pojedine vježbe ako im se bodovi potvrde na usmenom ispitivanju.

Usmeno ispitivanje traje otprilike 20 minuta i provodi se za sve studente određene grupe u isto vrijeme u nekom od laboratorija Fakulteta.
Na usmeno ispitivanje grupa treba donijeti računalo na kojemu će moći pokazati rezultate laboratorijske vježbe.
Ako grupa nije u mogućnosti donijeti svoje računalo, voditelj grupe treba se barem tjedan dana prije predaje javiti na službenu e-mail adresu predmeta kako bi se dogovorili uvjeti za predaju vježbe.

\textbf{Važna napomena:} Na svakoj od dvije usmene predaje svaki član grupe mora biti posebno dobro upoznat s barem jednim značajnim dijelom barem jedne od dvije laboratorijske vježbe koje se ispituju.
Neki primjeri značajnih dijelova bit će navedeni u uputama za laboratorijske vježbe.
Na usmenom ispitivanju pojedinačno se za svakog člana grupe potvrđuje ili poništava bodove dobivene računalnom evaluacijom programskog rješenja, ovisno o tome je li gornji uvjet zadovoljen.
Intencija ovog uvjeta je da svi članovi grupe aktivno sudjeluju u dizajnu i implementaciji laboratorijskih vježbi.
Nadalje, očekuje se da će studenti koji su poštovali navedeni uvjet vrlo lako ostvariti potvrdu bodova.

\textbf{I.~laboratorijska cjelina za predaju putem računala} \\
Sadržaj: Leksički analizator \\
Rok za predaju rješenja: 1.~11.~2015.

\textbf{II.~laboratorijska cjelina za predaju putem računala} \\
Sadržaj: Sintaksni analizator \\
Rok za predaju rješenja: 22.~11.~2015.

\textbf{I.~usmeno ispitivanje} \\
Sadržaj: rješenje zadataka I i II laboratorijske cjeline  \\
Termin predaje: 7.~12.~2015.~–- 11.~12.~2015.

\textbf{III.~laboratorijska cjelina za predaju putem računala} \\
Sadržaj: Semantički analizator \\
Rok za predaju rješenja: 20.~12.~2015.

\textbf{IV.~laboratorijska cjelina za predaju putem računala} \\
Sadržaj: Generator ciljnog programa. Optimiranje \\
Rok za predaju rješenja: 24.~1.~2016.

\textbf{II.~usmeno ispitivanje} \\
Sadržaj: rješenje zadataka III.~i IV.~laboratorijske cjeline  \\
Termin predaje: 25.~01.~2016.~- 29.~1.~2016.

\cleardoublepage  
\phantomsection  
\addcontentsline{toc}{section}{Sudjelovanje u nastavi}
\subsection*{Sudjelovanje u nastavi}

Studente se potiče na aktivno studjelovanje u nastavi te se takvo sudjelovanje dodatno boduje.
Primjerice, studente se potiče na sudjelovanje u diskusijama s nastavnikom tijekom predavanja (na svakom predavanju studenta se može usmeno ispitati), prijavu mogućih pogrešaka u nastavnim materijalima ili konkretnih konstruktivnih prijedloga za njihovo unapređenje (udžbenik, skripta sa zadacima, upute za laboratorijske vježbe i ove upute) te prijavu mogućih grešaka u računalnim sustavima koji se koriste za provedbu laboratorijskih vježbi.
Ako više studenata prijavi istu pogrešku ili prijedlog, mogući bodovi dodijeliti će se samo studentu koji je prvi prijavio pogrešku ili prijedlog.

Student sudjelovanjem u nastavi može sakupiti ukupno -15 $\leq$ U $\leq$ $\infty$ bodova.

\cleardoublepage  
\phantomsection  
\addcontentsline{toc}{section}{Kodeks ponašanja}
\subsection*{Kodeks ponašanja}

Od studenata se očekuje poštivanje kodeksa ponašanja Fakulteta elektrotehnike i računarstva.

Načelno, iako studente potičemo na zajedničku pripremu i suradnju u savladavanju gradiva tijekom semestra, od svakog se studenata očekuje samostalan rad u polaganju pojedinih obaveza predmeta.
Studenti \textbf{ne smiju} prikazati tuđi rad kao svoj te ne smiju primiti ili pružiti nedopuštene oblike pomoći tijekom pismenih provjera znanja (međuispiti, završni ispit, ispiti na ispitnim rokovima, kratke provjere znanja). 

Iznimno, studentima koji su se opredjelili za težu inačicu laboratorijskih vježbi, a pripadaju istoj grupi,  za laboratorijske vježbe dopuštena je i potiče se suradnja s ciljem zajedničkog rješavanja zadataka laboratorijskih vježbi.
Također, iako suradnja između dvije ili više grupa za laboratorijske vježbe u pogledu rasprave ideja vezanih za rješavanje zadataka laboratorijskih vježbi nije zabranjena, takva suradnja ne smije rezultirati potpuno istim ili vrlo sličnim programskim kodom.
Studente upozoravamo da ne pokušavaju prevariti sustav preuređivanjem tuđih programskih kodova, budući da se sličnost programskih kodova analizira i primjenom računala i primjenom ljudskih ispitivača.

Kazna za studente koji prekrše kodeks ponašanja jest 0 bodova iz provjere za koju se utvrdilo kršenje kodeksa te prijava disciplinskoj komisiji Fakulteta.

\cleardoublepage  
\phantomsection  
\addcontentsline{toc}{section}{Nadoknade}
\subsection*{Nadoknade}

Nadoknade međuispita ili završnog ispita nisu moguće ni pod kojim uvjetima.
Studenti koji zbog propuštenog međuispita ili završnog ispita ne sakupe dovoljan broj bodova za polaganje predmeta imaju priliku položiti predmet na ispitima ispitnih rokova.

Nadoknade predaja laboratorijskih vježbi putem računala također nisu moguće ni pod kojim uvjetima.

Nadoknade kratkih provjera znanja i usmenih predaja laboratorijskih vježbi moguće su isključivo iz opravdanih razloga te isključivo uz predočenje pismene dokumentacije koja potvrđuje opravdanost razloga izostanka.
Studentima koji nemaju pismenu dokumentaciju koja opravdava njihov izostanak neće biti omogućem pristup nadoknadama, bez izuzetaka.
Nadoknade za sve kratke provjere znanja bit će organizirane na kraju semestra.
Nadoknade za usmeno ispitivanje laboratorijskih vježbi bit će organizirane prema potrebi.

\cleardoublepage  
\phantomsection  
\addcontentsline{toc}{section}{Prenošenje bodova dobivenih prethodnih akademskih godina}
\subsection*{Prenošenje bodova dobivenih prethodnih akademskih godina}

Studentima koji su neuspješno polagali predmet prethodnih akademskih godina, ali su zadovoljili bodovni uvjet ili iz praktičnog dijela ili iz teoretskog dijela predmeta \textbf{ne omogućuje} se prenošenje navedenog dijela ostvarenih bodova na trenutno upisanu akademsku godinu.

\cleardoublepage  
\phantomsection  
\addcontentsline{toc}{section}{Ukupna ocjena}
\subsection*{Ukupna ocjena}

Ukupna ocjena u bodovima zbroj je svih bodova: BODOVI = KM + KZ + U + (L1 ili L2).
Uvjet za polaganje predmeta je L1 $\geq$ 10 ili L2 $\geq$ 20, odnosno sakupljanje barem 50\% bodova iz laboratorijskih vježbi, te KM + KZ + U $\geq$ 30, odnosno sakupljanje barem 50\% bodova iz teorijskog dijela predmeta. \\
Ukupna ocjena određuje se na osnovi ostvarenih BODOVA:
\begin{myindentpar}{30pt}
Dovoljan: 40 ili 50 bodova (ovisno o izboru lab.~vj.) \\
Dobar: 63 bodova \\
Vrlo dobar: 75 bodova \\
Izvrstan: 88 bodova 
\end{myindentpar}

Studenti koji ne ostvare dovoljan broj bodova za polaganje predmeta putem međuispita imaju pravo na polaganje predmeta putem ispita na ispitnim rokovima, ali samo ako su ostvarili prolaznost na laboratorijskim vježbama, odnosno ako su tijekom semestra sakupili najmanje 50\% bodova iz laboratorijskih vježbi.

Studentima se nude dvije inačice ispita, lakša inačica, za ocjenu dovoljan (2), te teža inačica za višu ocjenu.

Studentima koji pristupe ispitu na ispitnom roku ukupna ocjena u bodovima jednaka je bodovima ostvarenima na ispitu: BODOVI = I.
Bodovi koje student ostvari tijekom semestra ne prenose se na ispitne rokove.

Studenti koji su pristupili lakšoj inačici ispita dobivaju ocjenu dovoljan (2) ako su ostvarili barem 50\% bodova na ispitnom roku.

Ukupna ocjena na predmetu za studente koji pristupaju ispitnom roku za višu ocjenu određuje se na osnovi ostvarenih BODOVA:
\begin{myindentpar}{30pt}
Dovoljan: 50 bodova \\
Dobar: 63 bodova \\
Vrlo dobar: 75 bodova \\
Izvrstan: 88 bodova 
\end{myindentpar}

\end{document}
