\documentclass[times, 12pt, utf8]{book}
\RequirePackage[utf8]{inputenc}
\RequirePackage[croatian]{babel}
\usepackage[utf8]{inputenc}
\usepackage{amsmath}
\usepackage{multirow}
\usepackage{alltt}
\usepackage{listings}
\usepackage{color}
\usepackage{array}
\usepackage{hyperref}
\usepackage{verbatim}
\usepackage[a4paper, top=1.5cm, bottom=1.5cm, left=1.5cm, right=1.5cm]{geometry}
\pagestyle{plain}

\usepackage{titlesec}
\titleformat{\chapter}{}{\thechapter}{.01em}{}

\setlength{\parindent}{0pt}
\setlength{\parskip}{1ex plus 0.5ex minus 0.2ex}

\newenvironment{myindentpar}[1]%
{\begin{list}{}%
         {\setlength{\leftmargin}{#1}}%
         \item[]%
}
{\end{list}}

\addto\captionscroatian{%
  \renewcommand{\contentsname}%
    {}%
}

\titlespacing*{\chapter}{0pt}{50pt}{-60pt}

\title{
\vfill
\textbf{Prevođenje programskih jezika}\\
\vspace{30pt}
Informacije o predmetu za akademsku godinu 2012./2013. \\
\vspace{40pt}
}
\author{
\emph{Predavanja:} \\
Prof.~dr.~sc.~Siniša Srbljić (nositelj) \\
Dr.~sc.~Miroslav Popović \\
\vspace{30pt}
Dr.~sc.~Dejan Škvorc \\
\emph{Auditorne i laboratorijske vježbe:} \\
Zvonimir Pavlić, mag.~ing. \\
\vspace{30pt}
Ivan Budiselić, dipl.~ing. \\
\emph{Službena web-stranica predmeta:} \\
\vspace{30pt}
\url{http://www.fer.hr/predmet/ppj_a} \\
\emph{Službena e-mail adresa:} \\
\tt{\href{mailto:ppj@zemris.fer.hr}{ppj@zemris.fer.hr}}
}
\date{
\vspace{30pt} 
\emph{Datum posljednje izmjene:} \\
\today \\
\vfill
Fakultet elektrotehnike i računarstva \\
Zavod za elektroniku, mikroelektroniku, inteligentne i računalne sustave
}
\begin{document}
\maketitle

\section*{Informacije o predmetu}

Ovaj dokument sadrži osnovne informacije o organizaciji i provođenju predmeta Prevođenje programskih jezika tijekom akademske godine 2012./2013. 
Molimo studente da se upoznaju s informacijama u ovom dokumentu.

\let\cleardoublepage\relax
\phantomsection

\tableofcontents

\cleardoublepage  
\phantomsection  
\addcontentsline{toc}{section}{Literatura}  
\subsection*{Literatura}

\textbf{Osnovna literatura}

Srbljić: \textit{Prevođenje programskih jezika}, Udžbenik Sveučilišta u Zagrebu, Element, Zagreb, prvo izdanje 2007.

Srbljić: \textit{Jezični procesori 2: Analiza izvornog i sinteza ciljnog programa}, Udžbenik Sveučilišta u Zagrebu, Element, Zagreb, drugo izdanje 2003.

\textbf{Dodatna literatura}

Aho, Lam, Sethi, Ullman: \textit{Compilers: Principles, Techniques and Tools (Dragon book)}, Addison-Wesley, 2006. 

Appel, Palsberg: \textit{Modern Compiler Implementation in Java (Tiger book)}, Cambridge University Press, 2002.

Muchnick: \textit{Advanced Compiler Design and Implementation (Whale book)}, Morgan Kaufman Publishers, 1997.

Cooper, Torczon: \textit{Engineering a Compiler (Ark book)}, Morgan Kaufman Publishers, 2011.

Allen, Kennedy: \textit{Optimizing Compilers for Modern Architectures}, Morgan Kaufman Publishers, 2001.

Scott: \textit{Programming Languages Pragmatics}, Morgan Kaufmann Publishers, 2009.

\cleardoublepage  
\phantomsection
\addcontentsline{toc}{section}{Konzultacije i komunikacija s nastavnicima}
\subsection*{Konzultacije i komunikacija s nastavnicima}

Studente potičemo da komuniciraju s nastavnicima i rješavaju moguće probleme ili putem konzultacija ili slanjem poruke na službenu e-mail adresu predmeta.

Za konzultacije ne postoji zaseban termin, nego se održavaju po potrebi.
Konzultacije su moguće isključivo uz najavu barem jedan dan unaprijed s kratkim opisom problema na službenu e-mail adresu predmeta.

Nadalje, molimo studente da pitanja u vezi predmeta šalju isključivo na službenu e-mail adresu predmeta, a ne pojedinačno nastavnicima. 
Na poruke poslane izravno na e-mail adresu nekog od nastavnika u načelu se neće odgovarati.
Osim toga, na poruke s pitanjima na koje je odgovoreno u ovim uputama ili u obavijestima na službenoj web-stranici predmeta također neće biti poslani odgovori.

\begin{comment}
\cleardoublepage  
\phantomsection  
\addcontentsline{toc}{section}{Administracija bodova}
\subsection*{Administracija}

Za unos i pregled obaveza na predmetu i svih bodova koristi se informacijski sustav \href{http://ferko.fer.hr}{FERKO}.
Molimo studente da porukom na službenu e-mail adresu predmeta na vrijeme dojave moguće pogreške u unosu bodova.

\end{comment}

\cleardoublepage  
\phantomsection  
\addcontentsline{toc}{section}{Red predavanja}
\subsection*{Red predavanja}

Predavanja se u pravilu održavaju prema \href{http://web.zpr.fer.hr/satnica1/predmeti.htm}{službenoj satnici Fakulteta} i kalendaru nastave, sa po dva dvosatna predavanja tjedno.
Iznimno, između nekih predavanja će doći do pauze zbog održavanja auditornih vježbi u terminima predavanja ili zbog neradnih dana.

\textbf{01.~Predavanje} 
\begin{myindentpar}{30pt}
Osnovne faze procesa prevođenja: analiza i sinteza; Analiza izvornog programa: leksička, sintaksna i semantička analiza; Sinteza ciljnog programa: generiranje međukôda. (udžbenik, str.~1-15)
\end{myindentpar}

\textbf{02.~Predavanje} 
\begin{myindentpar}{30pt}
Sinteza ciljnog programa: optimiranje i generiranje ciljnog programa; Nadziranje pogrešaka tijekom prevođenja; Ocjena uspješnosti procesa prevođenja; Razredba jezičnih prevoditelja; (udžbenik, str.~15-31) 
\end{myindentpar}

\textbf{03.~Predavanje}
\begin{myindentpar}{30pt}
Leksička analiza; Uloga i podatkovna struktura leksičkog analizatora; Suradnja sa sintaksnim analizatorom; Nejednoznačnost u leksičkoj analizi; Prilagodba zapisa znakova, leksičke jedinke, leksičke pogreške i postupci oporavka od pogreške; (udžbenik, str.~44-55)
\end{myindentpar}

\textbf{04.~Predavanje}
\begin{myindentpar}{30pt}
Generatori leksičkog analizatora; Program Lex; (udžbenik, str.~56-70)
\end{myindentpar}

\textbf{05.~Predavanje}
\begin{myindentpar}{30pt}
Sintaksna analiza; Uloga i podatkovna struktura sintaksnog analizatora; Jezici za definiranje sintaksnih pravila; Jednostavni postupci parsiranja: parsiranje Co-No tablicom; (udžbenik, str.~71-84)
\end{myindentpar}

\textbf{06.~Predavanje}
\begin{myindentpar}{30pt}
Parsiranje od vrha prema dnu; S, Q i LL(1) gramatika; Određivanje PRIMIJENI skupova; (udžbenik, str.~84-103) 
\end{myindentpar}

\textbf{07.~Predavanje}
\begin{myindentpar}{30pt}
Određivanje PRIMIJENI skupova (nastavak); Prilagodba produkcija LL(1)-gramatici; Nadziranje pogrešaka; Parsiranje od dna prema vrhu; (udžbenik, str.~103-121)
\end{myindentpar}

\textbf{08.~Predavanje}
\begin{myindentpar}{30pt}
Tehnike parsiranja od dna prema vrhu: Pomakni-Pronađi, Pomakni-Reduciraj i prednost operatora; (udžbenik, str.~121-137) 
\end{myindentpar}

\textbf{09.~Predavanje}
\begin{myindentpar}{30pt}
LR parsiranje; (udžbenik, str.~138-147)
\end{myindentpar}

\textbf{10.~Predavanje}
\begin{myindentpar}{30pt}
LR parsiranje (nastavak); Generatori parsera; Program Yacc; Semantička analiza; Uloga i formalni modeli semantičkog analizatora; Zadaci semantičkog analizatora; (udžbenik, str.~147-166) 
\end{myindentpar}

\textbf{11.~Predavanje}
\begin{myindentpar}{30pt}
Zadaci semantičkog analizatora (nastavak); Sintaksom upravljana semantička analiza: atributna prijevodna gramatika; (udžbenik, str.~167-180)
\end{myindentpar}

\textbf{12.~Predavanje}
\begin{myindentpar}{30pt}
L-atributna prijevodna gramatika; Potisni automat za L-atributnu prijevodnu gramatiku; (udžbenik, str.~180-190)
\end{myindentpar}

\textbf{13.~Predavanje}
\begin{myindentpar}{30pt}
Potisni automat za L-atributnu prijevodnu gramatiku (nastavak); Metoda rekurzivnog spusta za L-atributnu prijevodnu gramatiku; Sustav obilježja; (udžbenik, str.~190-200) 
\end{myindentpar}

\textbf{14.~Predavanje}
\begin{myindentpar}{30pt}
Provjera vrijednosti obilježja; Jednakost vrijednosti obilježja; Potpora izvođenju ciljnog programa; (udžbenik, str.~200-223)
\end{myindentpar}

\textbf{15.~Predavanje}
\begin{myindentpar}{30pt}
Apstraktni tipovi podataka izvornog jezika i podatkovni objekti ciljnog jezika; Tijek izvođenja programa zasnovanog na procedurama; Organizacija i postupci dodjele memorije; (udžbenik, str.~223-233) 
\end{myindentpar}

\textbf{16.~Predavanje}
\begin{myindentpar}{30pt}
Pristup nelokalnim imenima; (udžbenik, str.~233-242)
\end{myindentpar}

\textbf{17.~Predavanje}
\begin{myindentpar}{30pt}
Razmjena ulazno/izlaznih parametara procedura; Generiranje međukoda; Razine i oblici međukôda; (udžbenik, str.~243-256) 
\end{myindentpar}

\textbf{18.~Predavanje}
\begin{myindentpar}{30pt}
Razine i oblici međukôda (nastavak); Sintaksom vođeno generiranje međukôda; Generiranje ciljnog programa; Struktura generatora ciljnog programa; (udžbenik, str.~256-265)
\end{myindentpar}

\textbf{19.~Predavanje}
\begin{myindentpar}{30pt}
Struktura generatora ciljnog programa (nastavak); (udžbenik, str.~265-276) 
\end{myindentpar}

\textbf{20.~Predavanje}
\begin{myindentpar}{30pt}
Algoritmi generatora ciljnog programa; Generatori generatora ciljnog programa; (udžbenik, str.~276-285) 
\end{myindentpar}

\textbf{21.~Predavanje}
\begin{myindentpar}{30pt}
Priprema ciljnog programa za izvođenje; Spremi-i-pokreni jezični procesori; Generatori izvodivog i premjestivog ciljnog programa; Generatori zasebnih dijelova programa; Program punitelj i program povezivač; Optimiranje; Analiza izvođenja programa; (udžbenik, str.~286-297)
\end{myindentpar}

\textbf{22.~Predavanje}
\begin{myindentpar}{30pt}
Analiza izvođenja programa (nastavak); (udžbenik, str. 297-305)
\end{myindentpar}

\textbf{23.~Predavanje}
\begin{myindentpar}{30pt}
Strojno zavisno i strojno nezavisno optimiranje; Jednostavni postupak optimiranja zasnovan na prozorčiću; Pregled ostalih postupaka optimiranja; (udžbenik, str.~305-317)
\end{myindentpar}

\cleardoublepage  
\phantomsection  
\addcontentsline{toc}{section}{Auditorne vježbe}
\subsection*{Auditorne vježbe}

Tijekom semestra, u terminima predavanja održavaju se i auditorne vježbe.
Auditorne vježbe namijenjene su za pripremu za međuispite i završni ispit, a u terminima auditornih vježbi rješavaju se primjeri tipičnih problemskih zadataka iz međuispita.

\textbf{I.~auditorne vježbe} \\
Tjedan održavanja: 3.~12.~2012.~–- 7.~12.~2012. \\
Gradivo: Leksička analiza, sintaksna analiza i semantička analiza

\textbf{II.~auditorne vježbe} \\
Tjedan održavanja: 21.~01.~2013.~–- 25.~01.~2013. \\
Gradivo: potpora izvođenju, generiranje i optimiranje ciljnog programa


\cleardoublepage  
\phantomsection  
\addcontentsline{toc}{section}{Međuispiti}
\subsection*{Međuispiti}

Tijekom semestra provode se pismene provjere znanja: međuispit i završni ispit.
Trajanje međuispita i završnog ispita je 120 minuta.
Međuispit i završni ispit sastoje se od 10 zadataka (5 teoretskih pitanja i 5 problemskih zadataka).

\textbf{Prvi međuispit} \\
Tjedan održavanja: 10.~12.~2012.~–- 14.~12.~2012. \\
Udio u ukupnoj ocjeni:  KM = 30 bodova \\
Gradivo: Leksička, sintaksna i semantička analiza

\textbf{Završni ispit} \\
Tjedan održavanja: 28.~1.~2013.~–- 8.~2.~2013. \\
Udio u ukupnoj ocjeni: KZ = 30 bodova \\
Gradivo: Ukupno gradivo ispredavano na predavanjima predmeta, odnosno ukupno gradivo udžbenika.

\cleardoublepage  
\phantomsection  
\addcontentsline{toc}{section}{Ispiti}
\subsection*{Ispiti}

Studenti koji ne polože predmet tijekom semestra putem međuispita i završnih ispita imaju pravo pristupa ispitnim rokovima ako su tijekom semestra ostvarili barem 12.5 bodova (50\%) iz laboratorijskih vježbi.

Ispitni rokovi sastoje se od 10 zadataka (5 teoretskih pitanja, 5 problemskih zadataka) i njihov udio u ukupnoj ocjeni je I = 100 bodova.
Drugim riječima, uspjeh na predmetu određen je uspjehom na ispitu.

\cleardoublepage  
\phantomsection  
\addcontentsline{toc}{section}{Kratke provjere znanja}
\subsection*{Kratke provjere znanja}

Tijekom semestra provode se i tri kratke provjere znanja putem Ahyco sustava u laboratorijima FER-a.
U svakoj se kratkoj provjeri u 10 zadataka (odabir između ponuđenih odgovora) ispituje znanje do tada ispredavanog gradiva.
Na kratkim provjerama znanja nema negativnih bodova za netočno riješene zadatke.
Za pristup kratkim provjerama znanja preporuča se korištenje preglednika Internet Explorer.
Provjera traje 10 minuta, a udio je ocjene jedne kratke provjere znanja u ukupnoj ocjeni 5 bodova.
Tri provjere ukupno nose N = 15 bodova.


I.~kratka provjera znanja: 5.~11.~2012.~–- 9.~11.~2012. \\
II.~kratka provjera znanja: 17.~12.~2012.~–- 21.~12.~2012. \\
III.~kratka provjera znanja: 21.~1.~2013.~–- 25.~1.~2013.


\cleardoublepage  
\phantomsection  
\addcontentsline{toc}{section}{Uvidi u rezultate pismenih provjera znanja}
\subsection*{Uvidi u rezultate pismenih provjera znanja}

Studenti imaju pravo uvida u rezultate pismenih provjera znanja, odnosno međuispita, završnog ispita, kratkih provjera znanja i ispita na ispitnim rokovima.
Nakon ispravljanja međuispita, završnih ispita i ispita na ispitnim rokovima na službenu web-stranicu predmeta bit će postavljena obavijest o službenom terminu uvida u rezultate.
Uvidi u rezultate bit će postavljeni u termin koji većini studenata stvara najmanje kolizija s ostalim fakultetskim obavezama, a uvidi će uvijek biti na ZEMRIS-u (3.~kat D zgrade).
Mole se studenti da ne ulaze u Zavod, nego da pričekaju ispred južnog ulaza Zavoda.
Studenti koji nisu u mogućnosti doći na uvid, ali isključivo iz opravdanog razloga, trebaju se porukom na službenu e-mail adresu predmeta najaviti s terminom u kojemu mogu doći na uvid, ali barem jedan dan prije odabranog termina za uvid.
Uvidi u rezultate kratkih provjera znanja mogući su također isključivo uz prethodnu najavu na službenu e-mail adresu predmeta, barem jedan dan prije predloženog termina.
Uvidi u rezultate izvan službenog termina za uvide ili bez prethodne najave na službenu e-mail adresu nisu mogući.

\cleardoublepage  
\phantomsection  
\addcontentsline{toc}{section}{Laboratorijske vježbe}
\subsection*{Laboratorijske vježbe}

Laboratorijske vježbe rade se u grupama od 6 studenata.
Studenti tijekom vježbi izrađuju vlastiti jezični procesor prema posebnim uputama za laboratorijske vježbe koje će biti objavljene na službenoj web-stranici predmeta.
Svaki student u grupi praktično izrađuje jedan dio jezičnog procesora prema dogovoru unutar grupe, ali mora biti upoznat sa svim dijelovima jezičnog procesora.
Grupu studenata vodi jedan od studenata (voditelj), a rad voditelja nadziru nastavnici.
Kao rezultat laboratorijskih vježbi studenti pokazuju rad svog jezičnog procesora.

Na početku semestra nastavnici definiraju koordinatore grupa za laboratorijske vježbe na osnovi rezultata iz predmeta ``Uvod u teoriju računarstva" protekle akademske godine.
Ulogu koordinatora grupe nije moguće odbiti.
Grupe se trebaju formirati tako da se studenti samostalno organiziraju i jave koordinatoru grupe.
U istoj grupi za laboratorijske vježbe mogu biti studenti različitih grupa za predavanja (primjerice, u istoj grupi za laboratorijske vježbe mogu biti studenti iz grupa za predavanja 3.~RZP1 i 3.~TIP1).
Voditelj grupe u pravilu je koordinator grupe, ali uz suglasnost koordinatora može biti i neki drugi član grupe, o čemu se studenti dogovaraju samostalno unutar grupe.
Nakon što formira grupu, isključivo koordinator dojavljuje nastavnicima JMBAG, ime i prezime svakog člana, odnosno voditelja grupe (potrebno je naznačiti koji je student voditelj) te u tom trenutku uloga koordinatora prestaje.
Drugim riječima, koordinator je potreban isključivo za formiranje grupe te određivanje voditelja.
Ukoliko koordinator ne uspije prikupiti dovoljan broj studenata u grupu, slučajnim odabirom u grupu će se dodati neraspodijeljeni studenti. 

Laboratorijske vježbe provode se kroz četiri ciklusa, a ocjenjivanje se provodi kombinacijom usmenog ispitivanja i računalne evaluacije studentskih rješenja.
Postoje četiri predaje laboratorijskih vježbi putem računala te dvije predaje usmenim ispitivanjem.

Usmeno ispitivanje traje otprilike 30 minuta i provodi se za sve studente određene grupe u isto vrijeme u nekom od laboratorija Fakulteta.
Na usmeno ispitivanje grupa treba donijeti računalo na kojemu će moći pokazati rezultate laboratorijske vježbe.
Ako grupa nije u mogućnosti donijeti svoje računalo, voditelj grupe treba se barem tjedan dana prije predaje javiti na službenu e-mail adresu predmeta kako bi se dogovorili uvjeti za predaju vježbe.
Na usmenom ispitivanju ocjenjuje se znanje studenata vezano uz dijelove jezičnog procesora koje je grupa programski ostvarila te gradivo predavanja vezano uz cjeline laboratorijskih vježbi.
Na usmenom ispitivanju studente se ocjenjuje pojedinačno; na svakom usmenom ispitivanju moguće je ostvariti najviše 5 bodova.

Predaje putem računala provode se izvan prostorija Fakulteta.
Za svaku predaju putem računala u računalni sustav dostupan putem Weba \textbf{isključivo} voditelj grupe dostavlja rješenje grupe za zadatak određen laboratorijskom vježbom.
Predano rješenje evaluira se automatizirano primjenom skupa ispitnih primjera.
Grupama čiji voditelji ne predaju programsko ostvarenje rješenja na vrijeme bit će oduzeti bodovi iz vježbe proporcionalni kašnjenju predaje rješenja.
Na osnovi evaluacije, svaka predaja boduje se s najviše 3.5 * broj\_studenata\_u\_grupi bodova.
Ostvareni broj bodova nastavnici objavljuju voditeljima grupe koji ukupnu sumu bodova raspodjeljuju članovima prema njihovim zaslugama u ostvarenju rješenja.
Pri tome, svaki član grupe ukupno na svim predajama putem računala od voditelja može dobiti najviše 5 bodova više od bodova koje je član ukupno dobio na svim usmenim ispitivanjima laboratorijskih vježbi.
Primjerice, ako je student iz svih usmenih ispitivanja dobio ukupno 7.5 od 10 bodova, onda od voditelja za predaju putem računala može dobiti ukupno najviše 7.5 + 5 = 12.5 od 15 bodova. 
Raspodjelu bodova prema članovima grupe voditelji dojavljuju nastavnicima predmeta, ali tek pred kraj semestra, nakon provedbe drugog usmenog ispitivanja.
Grupama čiji voditelji na vrijeme ne jave raspodjelu bodova automatski će se ravnomjerno raspodijeliti bodovi te će voditelju grupe biti dodijeljeno 0 bodova za vođenje grupe.

Dodatno, na drugom usmenom ispitivanju, voditeljima se dodjeljuje najviše 1 * broj\_studenata\_u\_grupi bodova za uspjeh u integraciji rješenja pojedinih ciklusa laboratorijskih vježbi u cjeloviti jezični procesor.
Navedeni bodovi također se smatraju bodovima dobivenim na predaji putem računala, odnosno navedene bodove za integraciju voditelj raspodjeljuje članovima grupe prema njihovim zaslugama.

Također, na drugom usmenom ispitivanju, isključivo voditeljima dodjeljuje se do 5 bodova za njihov uspjeh u vođenju grupe, pri čemu se 3 od 5 bodova određuje anonimnim glasanjem članova grupe, a 2 od 5 bodova određuju nastavnici.

Laboratorijske vježbe ukupno nose najviše L = 25 bodova za članove grupe, odnosno LV = 30 bodova za voditelje grupa.
Pri tome, kao što je opisano u prethodnim odlomcima, najviše 15 bodova od 25 studenti dobivaju od voditelja grupe za zasluge u ostvarenju rješenja zadataka laboratorijskih vježbi, a najviše 10 bodova dobivaju od nastavnika za znanje pokazano na usmenom ispitivanju.

\textbf{I.~laboratorijska cjelina za predaju putem računala} \\
Sadržaj: Leksički analizator \\
Rok za predaju rješenja: 28.~10.~2012.

\textbf{II.~laboratorijska cjelina za predaju putem računala} \\
Sadržaj: Sintaksni analizator \\
Rok za predaju rješenja: 18.~11.~2012.

\textbf{I.~usmeno ispitivanje} \\
Sadržaj: rješenje zadataka I i II laboratorijske cjeline te gradivo predavanja vezano uz I i II laboratorijsku cjelinu \\
Termin predaje: 3.~12.~2012.~–- 7.~12.~2012.

\textbf{III.~laboratorijska cjelina za predaju putem računala} \\
Sadržaj: Semantički analizator \\
Rok za predaju rješenja: 16.~12.~2012.

\textbf{IV.~laboratorijska cjelina za predaju putem računala} \\
Sadržaj: Generator ciljnog programa. Optimiranje \\
Rok za predaju rješenja: 20.~1.~2013.

\textbf{II.~usmeno ispitivanje} \\
Sadržaj: rješenje zadataka I., II., III.~i IV.~laboratorijske cjeline te gradivo predavanja vezano uz I., II., III.~i IV.~laboratorijsku cjelinu \\
Termin predaje: 21.~01.~2013.~- 25.~1.~2013.

\cleardoublepage  
\phantomsection  
\addcontentsline{toc}{section}{Sudjelovanje u nastavi}
\subsection*{Sudjelovanje u nastavi}

Studente se potiče na aktivno studjelovanje u nastavi te se takvo sudjelovanje dodatno boduje.
Primjerice, studente se potiče na sudjelovanje u diskusijama s nastavnikom tijekom predavanja (na svakom predavanju studenta se može usmeno ispitati), prijavu mogućih pogrešaka u nastavnim materijalima ili konkretnih konstruktivnih prijedloga za njihovo unapređenje (udžbenik, skripta sa zadacima, upute za laboratorijske vježbe i ove upute) te prijavu mogućih grešaka u računalnim sustavima koji se koriste za provedbu laboratorijskih vježbi.
Ako više studenata prijavi istu pogrešku ili prijedlog, mogući bodovi dodijeliti će se samo studentu koji je prvi prijavio pogrešku ili prijedlog.

Student sudjelovanjem u nastavi može sakupiti ukupno -15 $\leq$ U $\leq$ $\infty$ bodova.

\cleardoublepage  
\phantomsection  
\addcontentsline{toc}{section}{Kodeks ponašanja}
\subsection*{Kodeks ponašanja}

Od studenata se očekuje poštivanje kodeksa ponašanja Fakulteta elektrotehnike i računarstva.

Načelno, iako studente potičemo na zajedničku pripremu i suradnju u savladavanju gradiva tijekom semestra, od svakog se studenata očekuje samostalan rad u polaganju pojedinih obaveza predmeta.
Studenti \textbf{ne smiju} prikazati tuđi rad kao svoj te ne smiju primiti ili pružiti nedopuštene oblike pomoći tijekom pismenih provjera znanja (međuispiti, završni ispit, ispiti na ispitnim rokovima, kratke provjere znanja). 

Iznimno, studentima iste grupe za laboratorijske vježbe dopuštena je i potiče se suradnja s ciljem zajedničkog rješavanja zadataka laboratorijskih vježbi.
Također, iako suradnja između dvije ili više grupa za laboratorijske vježbe u pogledu rasprave ideja vezanih za rješavanje zadataka laboratorijskih vježbi nije zabranjena, takva suradnja ne smije rezultirati potpuno istim ili vrlo sličnim programskim kodom.
Studente upozoravamo da ne pokušavaju prevariti sustav preuređivanjem tuđih programskih kodova, budući da se sličnost programskih kodova analizira i primjenom računala i primjenom ljudskih ispitivača.

Kazna za studente koji prekrše kodeks ponašanja jest 0 bodova iz provjere za koju se utvrdilo kršenje kodeksa te prijava disciplinskoj komisiji Fakulteta.

\cleardoublepage  
\phantomsection  
\addcontentsline{toc}{section}{Nadoknade}
\subsection*{Nadoknade}

Nadoknade međuispita ili završnog ispita nisu moguće ni pod kojim uvjetima.
Studenti koji zbog propuštenog međuispita ili završnog ispita ne sakupe dovoljan broj bodova za polaganje predmeta imaju priliku položiti predmet na ispitima ispitnih rokova.

Nadoknade predaja laboratorijskih vježbi putem računala također nisu moguće ni pod kojim uvjetima.

Nadoknade kratkih provjera znanja i usmenih predaja laboratorijskih vježbi moguće su isključivo iz opravdanih razloga te isključivo uz predočenje pismene dokumentacije koja potvrđuje opravdanost razloga izostanka.
Studentima koji nemaju pismenu dokumentaciju koja opravdava njihov izostanak neće biti omogućem pristup nadoknadama, bez izuzetaka.
Nadoknade za sve kratke provjere znanja bit će organizirane na kraju semestra.
Nadoknade za usmeno ispitivanje laboratorijskih vježbi bit će organizirane prema potrebi.

\cleardoublepage  
\phantomsection  
\addcontentsline{toc}{section}{Prenošenje bodova dobivenih prethodnih akademskih godina}
\subsection*{Prenošenje bodova dobivenih prethodnih akademskih godina}

Studentima koji su neuspješno polagali predmet prethodnih akademskih godina, ali su zadovoljili bodovni uvjet ili iz praktičnog dijela ili iz teoretskog dijela predmeta \textbf{ne omogućuje} se prenošenje navedenog dijela ostvarenih bodova na trenutno upisanu akademsku godinu.

\cleardoublepage  
\phantomsection  
\addcontentsline{toc}{section}{Ukupna ocjena}
\subsection*{Ukupna ocjena}

Ukupna ocjena u bodovima zbroj je svih bodova: BODOVI = KM + KZ + N + U + (L ili LV).
Uvjet za polaganje predmeta je (L ili LV) $\geq$ 12.5, odnosno sakupljanje barem 50\% bodova iz laboratorijskih vježbi, te KM + KZ + N + U $\geq$ 25, odnosno sakupljanje barem 33.3\% bodova iz teorijskog dijela predmeta. \\
Ukupna ocjena određuje se na osnovi izračunatih BODOVA:
\begin{myindentpar}{30pt}
Dovoljan: 15\% pozitivno ocijenjenih studenata \\
Dobar: 45\% pozitivno ocijenjenih studenata \\
Vrlo dobar: 25\% pozitivno ocijenjenih studenata \\
Izvrstan: 15\% pozitivno ocijenjenih studenata  
\end{myindentpar}

Studenti koji ne ostvare dovoljan broj bodova za polaganje predmeta putem međuispita imaju pravo na polaganje predmeta putem ispita na ispitnim rokovima, ali samo ako su ostvarili prolaznost na laboratorijskim vježbama, odnosno ako su tijekom semestra sakupili najmanje 12.5 bodova iz laboratorijskih vježbi.

Studentima koji pristupe ispitu na ispitnom roku ukupna ocjena u bodovima jednaka je bodovima ostvarenima na ispitu: BODOVI = I.
Bodovi koje student ostvari tijekom semestra ne prenose se na ispitne rokove.
Ukupna ocjena na predmetu za studente koji pristupaju ispitnom roku određuje se na osnovi ostvarenih BODOVA:
\begin{myindentpar}{30pt}
Dovoljan: 50 bodova \\
Dobar: 63 bodova \\
Vrlo dobar: 75 bodova \\
Izvrstan: 88 bodova 
\end{myindentpar}

\end{document}
